\documentclass[aps, prc, reprint, amsmath, groupedaddress, nofootinbib]{revtex4-1}
%\usepackage[compat=1.1.0]{tikz-feynman}
\usepackage[utf8]{inputenc}
\usepackage{hyperref}
\usepackage{amsmath}
\usepackage{amssymb}
\usepackage{amsfonts}
\usepackage{tabularx}
\usepackage{booktabs}
\usepackage{graphicx}
\usepackage{color}
\usepackage{multirow}
\usepackage{verbatim}
\usepackage[inline]{enumitem}
\graphicspath{{fig/}}
\definecolor{theblue}{RGB}{0,50,230}
\usepackage{appendix}
\hypersetup{
  colorlinks=true,
  linkcolor=theblue,
  citecolor=theblue,
  urlcolor=theblue
} 


\begin{abstract}
A Monte-Carlo simulation is a useful tool in the phenomenology study of the interactions between hard partons and the quark-gluon plasma.
However the medium induced gluon radiation process--import at high energy--cannot be factorized into independent processes due to the coherence effect and is hard to implement in a Monte-Carlo way.
Many existing implementations capture qualitatively behaviors or only work in certain limits, but it is important to study if they quantitatively agree with the current knowledge from theory.
In this work, we compared several approaches with different physical motivations.
In the case of fixed coupling and static medium, we found that a particular implementation can be tuned to reproduce very well the semi-analytic calculations of the parton radiative energy loss.
The impacts of an expanding medium are also discussed.
This implementation is not only a good surrogate model of the existing theory, but also a practical one that can be coupled to realistic QGP medium evolution and tuned to experimental data. 
This way, we can reduce the modeling uncertainty in the Monte Carlo simulations and extract the transport properties of energetic partons inside a quark-gluon plasma in a more meaningful way in the future.


\end{abstract}


\begin{document}
\title{Compare Monte-Carlo implementations of parton radiative energy loss to theoretical limits.}
\author{Weiyao Ke}
\author{Steffen A.\ Bass}
\affiliation{Department of Physics, Duke University, Durham, NC 27708-0305}
\date{\today}
\maketitle

\section{Introduction}
In a perturbative picture, high energy partons lose energy to a quark-gluon plasma (QGP) medium mainly through the medium induced gluon radiation process.
For such partons, the Landau-Pomeranchuk-Migdal (LPM) effect significantly suppresses gluon radiation rate for those whose formation times ($\tau_f$) become comparable to or even larger than to their mean-free-path ($\lambda_g$).
For a gluon to be decoherent from its mother parton, the time it takes is of order,
\begin{eqnarray}
\tau_f = \frac{2\omega}{k_\perp^2+(1-x)m_g^2},
\end{eqnarray}
where $m_g^2=m_D^2/2 \sim \alpha_s T$ is gluon asymptotic mass squared.
For a collinear gluon the formation time looks like $\omega/(\alpha_s T^2)$.
Meanwhile, the gluon can undergoes multiple scatterings with the medium.
In a perturbative picture, this collision rate $R_{g} = 1/\lambda_g$ scales like $\alpha_s T$. 
Therefore, an estimate of the number of rescatterings within the formation time $N \sim \tau_f R_g \sim \omega/T$ may not be a small number for gluon with energy comparable or larger than the medium temperature.
What is more complicated is that each rescattering can change the transverse momentum of the gluon relative to the mother parton.
Therefore we need a self-consistent estimation of the formation time.
Given that on average, each rescattering broadens the gluon transverse momentum square $k_t^2$ by the amount $\hat{q}_g\lambda_g$ where $\hat{q}_g = d\langle k_t^2\rangle/dt$ is the gluon transport parameter, the self consistent relation reads,
\begin{eqnarray}
\tau_f &=& \frac{\omega}{\hat{q}_g\tau_f},\\
\tau_f &=& \sqrt{\frac{\omega}{\hat{q}_g}}
\end{eqnarray}
The physical picture of how the in-medium transport of gluon modify the energy differential rate of medium induced gluon radiation is nicely summarized as follows,
\begin{eqnarray}
\frac{dP}{dt d\omega} \sim \begin{cases}
 \frac{\alpha_s C_R}{\omega} \frac{1}{\lambda_g} \sim \alpha_s^2 C_R \frac{T}{\omega}, \hfill \tau_f < \lambda_g\\
 \frac{\alpha_s C_R}{\omega} \frac{1}{\tau_f}\sim \alpha_s C_R \sqrt{\frac{\hat{q}_g}{T^3}} \left(\frac{T}{\omega}\right)^{3/2}, \hfill \lambda_g < \tau_f
\end{cases}
\end{eqnarray}
This is understand as a gluon with energy $\omega$ first being split from the mother parton with the probability given by the vacuum splitting function ($\sim \alpha_s C_R/\omega$), and then be put on shell with the rate $1/\lambda_g$ if its formation time is smaller than the mean-free-path, otherwise multiple rescattering works coherently to put it onshell with a rate $1/\tau_f$. 
We note first that, it is a modification of single gluon emission rate, and second, on average, it suppresses the amount of radiation.

and also leads to a non-linear path-length dependence of the energy loss.
Though this effect has been studied extensively at the leading order, whether the various Monte-Carlo implementations results in quantitatively the same suppression is not clear. 
Moreover, one often implements the formula calculated for a static medium with a constant temperature, but the realistic medium in a heavy-ion collision undergoes fast expansion where temperature changes significantly within a time scale comparable to the LPM effect.
In this work, we studied three different Monte-Carlo implementations and systematically compare them to the semi-analytic theoretical limits.
We showed that by fine-tuning the parameters of one of the methods, the theoretical calculations are well reproduced at different coupling constants, parton energies, temperatures and path-lengths and this method generalizes to expanding medium. 
By showing that a Monte-Carlo implementation reduces to known theoretical calculation, a more meaning full theory-to-data comparison can be performed where the Monte-Carlo model works as a bridge between certain theory and experimental measurements.



\begin{acknowledgments}
SAB and WK  are supported by the U.S. Department of Energy Grant no. DE-FG02-05ER41367. WK is also supported by NSF grant OAC-1550225.
\end{acknowledgments}

\begin{appendices}
\end{appendices}
\bibliography{mclpm} 
\end{document}